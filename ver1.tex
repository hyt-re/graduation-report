\documentclass{jreport}
% 図を取り込むためのパッケージ
\usepackage[dvipdfmx]{graphicx}

% 必要に応じてコメントアウトを外して使用するパッケージ
%\usepackage{amsmath,amssymb} % 数式用
%\usepackage{bm}              % 太字ベクトル用
%\usepackage{url}             % URL参照用

% 【重要】学科指定のマクロ(このファイルと同じ場所に kcctd_paper.sty が必要)
\usepackage{kcctd_paper}

%%%%% 論文情報入力エリア %%%%%
\title{ここに論文タイトルを記入}  % タイトル
\author{吉田 隼人}               % 報告者(自分の名前)
\adviser{尾山 匡浩}              % 指導教員名
\date{2025年2月3日}              % 提出日
\year{7}                        % 令和○年度

%%%%% 論文要旨 (Abstract) %%%%%
\abstract{
ここに論文要旨を記述してください(1ページ以内)。
研究の背景、目的、方法、得られた結果、結論などを簡潔にまとめて記述します。
読者が本文をすべて読まなくても、概要が把握できるように書くことが重要です。
}
%%%%%%%%%%%%%%%%%%%%%%%%%%%%%%

\begin{document}

%--- 表紙・目次設定 ---
\pagenumbering{roman} % ページ番号をローマ数字(i, ii...)に設定
\maketitle            % 表紙と要旨の出力
\setcounter{page}{1}  % ページ番号の初期化
\tableofcontents      % 目次の出力

%--- 本文開始 ---
\chapter{序論}
\pagenumbering{arabic} % ここからページ番号をアラビア数字(1, 2...)に設定
% \setcounter{page}{1} % 必要であればここでページ番号を1にリセット

ここに第1章の本文を記述します。
研究の背景や目的について述べます。

\chapter{実験方法(または理論)}
ここに第2章の本文を記述します。
図を入れる場合は以下のように書きます。

% 図の挿入例
% \begin{figure}[htbp]
%   \centering
%   \includegraphics[width=8cm]{fig1.png}
%   \caption{図の説明}
%   \label{fig:example}
% \end{figure}

\chapter{実験結果}
実験や解析の結果を記述します。

\chapter{考察}
結果に対する考察を記述します。

\chapter{結論}
本研究のまとめと今後の課題などを記述します。

%--- 謝辞 ---
\chapter*{謝辞}
\addcontentsline{toc}{chapter}{謝辞} % 目次に追加
本研究を進めるにあたり、ご指導いただいた○○先生に深く感謝いたします。
また、実験に協力してくれた○○君に感謝します。

%--- 参考文献 ---
\begin{thebibliography}{99}
\addcontentsline{toc}{chapter}{参考文献} % 目次に追加

% 参考文献の書き方例
\bibitem{ref-1}
著者名: ``文献タイトル'', 出版社, pp.xx-xx (発行年).

\bibitem{ref-2}
著者名: ``論文タイトル'', 学会誌名, Vol.xx, No.xx, pp.xx-xx (発行年).

\end{thebibliography}

%--- 付録 (必要な場合のみコメントアウトを外す) ---
%\appendix
%\chapter{付録データの詳細}
%ここに付録などを記述します。

\end{document}